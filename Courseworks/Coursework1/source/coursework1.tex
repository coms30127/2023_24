\documentclass[12pt]{article}
\usepackage{amsfonts, epsfig}
\usepackage{amsmath}
\usepackage{graphicx}
\usepackage{fancyhdr}
\pagestyle{fancy}
\lfoot{\texttt{coms30127.github.io}}
\lhead{Computational Neuroscience - Coursework 1 - auto-associative networks}
\rhead{\thepage}
\cfoot{}
\begin{document}

\section*{Coursework 1}

This coursework relates to the Hopfield network. This coursework may
appear before we have covered Hopfield networks in the lectures, but
you can still get started!

In a Hopfield network you have a network of
neurons; these neurons can be in an up state:
\begin{equation}
  x=1
\end{equation}
or a down state
\begin{equation}
  x=-1
\end{equation}
In the synchronous version we will use here all the neurons are
updated at the same time; I will put a hat on the new value, obviously
in the nex iteration the new value becomes the current value and loses
the hat. Here there is no threshold and the update rule is:
\begin{equation}
  \hat{x}_i=\left\{\begin{array}{ll}1&\sum_j w_{ij}x_j\\-1&\text{otherwise}\end{array}\right.
\end{equation}

You can store patterns in a Hopfield network by setting the synapse
strengths, that is the $w_{ij}$'s. The more brain-like version has a
learning rule for changing $w_{ij}$; here we will anticipate the end
result of that learning and stipulate that
\begin{equation}
w_{ij}=\frac{1}{N}\sum_a x_i^a x_j^a
\end{equation}
where for each of $N$ values of the index $a$ the $x_i^a$'s represent
the patterns we wish the network to store. The idea is that after the
synapses have been set, that is after the network has learned, if a
partial patter is presented, so some but not all of the values of
$x_i^a$ for one of the $a$'s, the other $x_i^a$ will adjust to
complete the pattern.



In the \texttt{coursework1} folder you will find programmes
\texttt{seven\_segment.jl} and \texttt{seven\_segment.py}. These
contain a function for converting an 11-component vector of ones and
-1's into an old-fashioned seven-segment
digit\footnote{\texttt{https://en.wikipedia.org/wiki/Seven-segment\_display}}
and a number: the first seven digits correspond to the seven-segment
display and the remaining four code for the number in a sort of binary
where the zeros have been replaced with -1. It also contains three
patterns: one, three and six.

The goal of this coursework is to extend one or other of these
programmes to include a Hopfield network to store these three
patterns using the formula for $w_{ij}$ when $i\not=j$
\begin{equation}
w_{ij}=\frac{1}{N}\sum_a x_i^a x_j^a
\end{equation}
where $a$ labels patterns, $x_i^a$ is the activiation of the $i$th
node in the $a$th pattern and $N$ is the number of
patterns. $w_{ii}=0$.

In the code there are also two test patterns, your programme should
update these synchronously until convergence and print out the
patterns at each iteration. 


This coursework is intended to check you understand Hopfield networks,
you are not being asked to do anything over elaborate beyond that. In
summary you should:
\begin{itemize}
\item Create a weight matrix of $w_{ij}$s.
\item Fix the weight values to store the three patterns.
\item Write a function to evolve the network according to the
  McCulloch-Pitts formula; this should be done synchronously so all the nodes are updated at each timestep. Assume the threshold $\theta=0$.
\item For each of the two test patterns, evolve the patterns until they stop changing, printing the pattern at each step.
\end{itemize}

Add to the above a function to calculate the energy of a configuration
\begin{equation}
E=-\frac{1}{2}\sum_{ij} x_i w_{ij} x_j
\end{equation}
and print out the energy of the three learned patterns, of the test
patterns and any patterns formed as the patterns are updated.

\subsection*{Submission instructions}

This coursework used to be for credit and so there is a submission
script, if you are curious you can still look at it, it generated an
automatic report which was then uploaded, instructions are in
\texttt{submission.README}. While this coursework is no longer marked,
I do, as usual, offer a small prize for anyone who does the project in
a genuinely odd choice of programming language.

\end{document}
