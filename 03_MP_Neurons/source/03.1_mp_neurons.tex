\documentclass[12pt]{article}
\usepackage{amsfonts, epsfig}

\usepackage{graphicx}
\usepackage{fancyhdr}
\pagestyle{fancy}
\lfoot{\texttt{coms30127.github.io}}
\lhead{Computation Neuroscience - 03.1\_mp\_neurons - Conor}
\rhead{\thepage}
\cfoot{}

\usepackage{graphicx}
\usepackage{pstricks}
\usepackage{listings}

\usepackage{tikz}

\usepackage{pgf}
\usepackage[utf8]{inputenc}
\usetikzlibrary{arrows,automata}
\usetikzlibrary{positioning}


\tikzset{
    state/.style={
           rectangle,
           rounded corners,
           draw=black, very thick,
           inner sep=2pt,
           text centered,
           },
}
\tikzset{
    on/.style={
           circle,
           draw=red, very thick,
           inner sep=2pt,
           fill=red!25,
           },
}


\tikzset{
    off/.style={
           circle,
           draw=blue, very thick,
           inner sep=2pt,
           text centered,
           },
}



\tikzset{
    neuron/.style={
           rectangle,
           rounded corners,
           draw=black, very thick,
           inner sep=2pt,
           text centered,
           },
}



\tikzset{
    area/.style={
           rectangle,
           draw=black, very thick,
           inner sep=2pt,
           text centered,
           },
}


\tikzset{
    gc/.style={
           rectangle,
           rounded corners,
           draw=red, very thick,
           inner sep=2pt,
           text centered,
           },
}


\tikzset{
    inh/.style={
           rectangle,
           rounded corners,
           draw=blue, very thick,
           inner sep=2pt,
           text centered,
           },
}



\tikzset{
    io/.style={
           rectangle,
           draw=green, very thick,
           inner sep=2pt,
           text centered,
           },
}



\begin{document}

\section*{McCulloch-Pitts neurons} 

The McCulloch Pitts neuron model, or Threshold Logic Unit, was
introduced in 1943 by Warren McCulloch and Walter
Pitts\footnote{Walter Pitts was an interesting and odd man, a genius
in the old-fashioned self-destructive and brilliant sense.} as a
computational model of a neuronal network
\cite{McCullochPitts1943}. Their thinking was that neurons are joined
to each other with connections of variable strength; in the soma the
inputs from other neurons are added up and they determine the activity
of the neuron in a non-linear way; they also knew that neurons tend to
ignore input up to some threshold value before responding
strongly. These properties they tried to include in their model
neurons.

Artificial neurons, of the sort used in artificial intelligence, are
described by a single dynamical variable, $x_i$ say for a neuron
labelled $i$; the value of $x_i$ is determined by the weighted input
from the other neurons:
\begin{equation}
x_i=\phi\left(\sum_j w_{ij} x_j-\theta_i\right)
\end{equation}
$\phi$ is an activation function, $\theta$ is a threshold and the
$w_{ij}$ are the connection strengths weighting the inputs from the
other neurons. The McCulloch Pitts neuron was the first example of an
artificial neuron and had a step function for $\phi$:
\begin{equation}
x_i=\left\{\begin{array}{ll}1&\sum_j w_{ij} x_j>\theta_i\\-1&\mbox{otherwise}\end{array}\right.
\end{equation}
Thus, the neuron has two states, it is in the on state, $x_i=1$ if the
weighted input exceeds a threshold $\theta_i$ and an off state,
$x_i=-1$ if it doesn't; the picture you might have of how this
corresponds to the brain is that \lq{}on\rq{} corresponds to rapid
spiking and \lq{}off\rq{} to spiking at a much lower rate. The
$w_{ij}$, the connection strengths, are like the synapse strengths, a
positive $w_{ij}$ is an excitatory synapse and negative, an
inhibitory; a given neurons have both negative and positive
out-going synapses, that is there is no restriction that says $w_{ij}$
always has the same sign for a given $j$, this is different from real
neurons where all the outgoing synapses from a given neuron are either
excitatory or inhibitory.

While it should be clear that this network has some of the properties,
very abstracted, of a neuronal network, it might not be so clear what
can be done with the neurons. When they were working, at the dawn of
the age of electronic computers, McCulloch and Pitts believed that
their neurons might form the natural unit in computer circuits. In
other words, they thought they might perform the role actually played
by logical circuits. In fact, it is still not clear if the artificial
neuron is or isn't the natural unit of computation since they are a
component in, for example, deep learning networks. In fact, there are
two major applications of McCulloch-Pitts neurons: the perceptron and
the Hopfield network. These two applications add a rule for changing
the connection strength to the original McCulloch-Pitts neuron.

\begin{thebibliography}{10}

\bibitem{McCullochPitts1943}
McCulloch, W and Pitts, W. (1943). A logical calculus of the ideas immanent in nervous activity. 
\newblock Bulletin of Mathematical Biophysics, 5:115--133. 

\end{thebibliography}

\end{document}
