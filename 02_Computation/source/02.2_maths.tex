\documentclass{article}
\usepackage[utf8]{inputenc}
\usepackage{graphicx}
\usepackage{fancyhdr}
\usepackage{epsfig}
\usepackage{epstopdf}
\usepackage{pstricks}

\usepackage{ifthen}
\newboolean{nopics}
\setboolean{nopics}{false}


\pagestyle{fancy}
\lfoot{\texttt{coms30127.github.io}}
\lhead{Computation Neuroscience - 02.2\_maths - Conor}
\rhead{\thepage}
\cfoot{}
\begin{document}
\section*{Some mathematics}
\subsection*{A first order differential equation}
Many of the differential equations we come across in neuroscience are
inhomogeneous first order differential equations
\begin{equation}
\tau\frac{df}{dt}=g(t)-f(t)
\end{equation}
where $f(t)$ is some function of $t$ we are interested, $g(t)$ is some
outside \textsl{driving function} and $\tau$ is a constant. One
example we will examine is the integrate and fire equation:
\begin{equation}
\tau_m\frac{dV}{dt}=E_l-V(t)+g_lI(t)
\end{equation}
where $V(t)$ is the voltage inside a neuron, and the driving function
is made up of two parts $E_l$, the so called reversal potential and
$g_lI(t)$ which represents current coming into the cell, for example,
through an electrode if the cell being modelled is being experimented
on, or because of signals from other neurons. Another example is the
gating equation which represents the opening and closing of
ion-channels, little gates in the membrane of the cells and a third
example is the equation for synaptic conductance. In each of these
examples the underlying equation is the same, or nearly the same, and
so we will look at it now in the abstract without considering the
application.


These are called first order linear differential equations, first
order because the highest derivative is $df/dt$, there is no
$d^2f/dt^2$ term for example, and linear because there are no
non-linear $f$ terms, no $f^2$ or anything like that. First order
linear differential equations can be solved, or, at least the solution
can be written as an integral. The solution uses an
\textsl{integrating factor}: the trick of simplifying the equation by
multipling across by
\begin{equation}
\lambda=e^{t/\tau}
\end{equation}
This gives
\begin{equation}
\tau e^{t/\tau}\frac{df}{dt}+e^{t/\tau}f=e^{t/\tau}g(t)
\end{equation}
The clever bit, now, is that the two terms on the left hand side can
be rewritten in product form
\begin{equation}
\tau \frac{d}{dt}\left(e^{t/\tau}f\right)=e^{t/\tau}g(t)
\end{equation}
We integrate both sides
\begin{equation}
\tau e^{t/\tau}f(t)=\int_0^t e^{t'/\tau} g(t')dt'+A
\end{equation}
where $A$ is some integration constant. Setting $t=0$ shows that $A=\tau f(0)$ and, hence, dividing across by the stuff in front of $f(t)$
\begin{equation}
f(t)=e^{-t/\tau}\left[\frac{1}{\tau}\int_0^t e^{t'/\tau} g(t')dt'+f(0)\right]
\end{equation}
Hence, $f(t)$ is written as an integral, if we can do the integral we
have solved the equation.

The easiest case is obviously $g(t)=\bar{g}$ a constant:
\begin{equation}
f(t)= \bar{g} \left[\frac{1}{\tau}e^{-t/\tau}\int_0^t e^{t'/\tau}dt'\right]+f(0)e^{-t/\tau}=\bar{g}\left(1-e^{-t/\tau}\right) + f(0)e^{-t/\tau}
\end{equation}
or
\begin{equation}
f(t)=\bar{g}+[f(0)-\bar{g}]e^{-t/\tau}
\end{equation}
It is easy to understand this solution: 
\begin{equation}
\tau\frac{df}{dt}=\bar{g}-f(t)
\end{equation}
has $df/dt=0$ only if $f(t)=\bar{g}$ furthermore if $f(t)>\bar{g}$
then $df/dt<0$ and $f(t)$ decreases towards $\bar{g}$; if
$f(t)<\bar{g}$ so $df/dt>0$ then $f(t)$ increases. Since $df/dt$ is zero
for $f(t)=\bar{g}$ we say this is the equilibrium; since the sign of
$df/dt$ means that $f(t)$ always approaches $\bar{g}$ we say this
solution is stable.


We can see in the
solution that $f(t)$ decays exponentially towards $\bar{g}$ and it
does so with a time scale that depends on $\tau$. An example is shown in Fig.~\ref{exp_decay}.

\begin{figure}[tb]
\ifthenelse{\boolean{nopics}}
{\textsl{In a graph with vertical range -0.5 to 2.5 and horizontal range zero to 20, there is a horizontal line at zero; two curves approach it as you go from left to right but they become more and more horizontal the closer they get, the top curve starts at 2.5, the bottom from -0.5 so the top curve falls quicker than the bottom rises.}}
{
  \begin{center}
%\includegraphics{exp_decay}
  \end{center}
  }
\caption{Two examples of $f(t)$ decaying towards $\bar{g}$. Here
  $\bar{g}=1$ and $\tau=10$; in the top curve $f(0)=2.5$ whereas in
  the other $f(0)=0.5$. \label{exp_decay}}
\end{figure}

What happens if $g(t)$ isn't constant; well some of the same logic applies, the sign of $df/dt$ means that the solution is trying to get to $g(t)$ but the difference is now that $g(t)$ might change before it gets there. Actually looking at 
\begin{equation}
f(t)=e^{-t/\tau}\left[\frac{1}{\tau}\int_0^t e^{t'/\tau} g(t')dt'+f(0)\right]
\end{equation}
lets consider the function
\begin{equation}
T(t)=\frac{1}{\tau}e^{-t/\tau}\int_0^t e^{t'/\tau} g(t')dt'
\end{equation}
and do a change of variable $t'=t-s$ so 
\begin{equation}
T(t)=\frac{1}{\tau}\int_0^t e^{-s/\tau} g(t-s)ds
\end{equation}
then the role of the integral is to filter or smooth out $g(t)$ by
averaging it over its past with a exponential window. This is
illustrated in Fig.~\ref{filtering}.

\begin{figure}[tb]
\ifthenelse{\boolean{nopics}}
{\textsl{In this picture there are two `curves', one is piecewise constant, giving a blocky plot, the other curves exponentially so that it tries to equal the first. When the blocky curve rises to one the other curve rises slowly to meet it, the blocky curve then falls to zero and the other one falls again, next the block curve rises to two and the other curve goes up more quickly, the blocky curve falls again to zero and the other curve follows. Finally the block curve oscillates rapidly between a constant value of one and a value of zero, the other curve moves jaggedly in a more restricted range since it neither has time to approach one during the block curves up sections, or to reach near zero during its down sections. }}
{
  \begin{center}
%\includegrahics{filtering}
  \end{center}
  }
\caption{This shows how the response $f(t)$ filters the input $g(t)$;
  the differntial equation has $\tau=3$ and an input $g(t)$ which
  varies as shown above; the code used to calculate $f(t)$ is
  \texttt{filtered\_input.py}. \label{filtering}}
\end{figure}

Imagine that $g(t)$ varies slowly compared to the time scale $\tau$, so by the time $g(t-s)$ changes much compared to $g(t)$ then $\exp(-s/\tau)$ is more-or-less zero, then
\begin{equation}
T(t)\approx\frac{1}{\tau}g(t)\int_0^t e^{-s/\tau}ds=g(t)\left(1-e^{-t/\tau}\right)
\end{equation}
and substituting back in we get
\begin{equation}
f(t)\approx g(t)+[f(0)-g(t)]e^{-t/\tau}
\end{equation}

\end{document}
