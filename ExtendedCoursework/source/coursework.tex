\documentclass[12pt]{article}
\usepackage{amsfonts, epsfig}

\usepackage{graphicx}
\usepackage{fancyhdr}
\pagestyle{fancy}
\lfoot{\texttt{coms30127.github.io}}
\lhead{Computation Neuroscience - Extended Coursework - spike trains}
\rhead{\thepage}
\cfoot{}
\begin{document}

\section*{Coursework}

This coursework relates to the properties of spike train. These
questions are adapted from the exercises given in Part 1.1 of\\
\\
\texttt{http://www.gatsby.ucl.ac.uk/$\sim$dayan/book/exercises.html}\\
\\
in their original form they are exercises for the book Dayan and
Abbott, a recommended text book for this course. In these question we
are interested in the properties that spike trains have; this gives a
more data-led perspective on spike trains than the one we have looked
at before.

\subsection*{Poisson processes}

The first question relates to Poisson processes, a Poisson process
is a memoryless random process producing a time series of events; in
the application to spike trains an event corresponds to a spike. By
memoryless it is meant that the chance of an event depends only on a
rate function $r(t)$ and not on the history of past events. The idea
is that the probability of an event in a small interval $\delta t$ wide is $r(t)\delta t$.

Imagine a person fishing in the sea, the chance of catching a fish
might depend on the time of day but it doesn't depend on how many
fishes the person has already caught, however, if they are fishing in
a small pond the chance of catching a fish would diminish as they
catch fish, so fishing in the sea is a Poisson process, but fishing in
a pond is not. In the questions here the Poisson process is
homogenous, that is, the rate is constant. Of course, if there is a
refractory period, that is a time during which the neuron cannot
spike, the spike train is not a Poisson process, but in this question
a Poisson process is used to generate the spike train.

It can be proved, with a nice argument you are urged to look up, that
the distribution of inter-event intervals $t$, here inter-spike intervals,
is given by
\begin{equation}
p(t)=\frac{1}{r}e^{-rt}
\end{equation}
A distinctive feature of a Poisson process is that its Fano factor and
coefficient of variation. The aim here is to calculate these
quantities for simulated and real spike trains.

The Fano factor is defined as
\begin{equation}
F=\frac{\sigma^2}{\mu}
\end{equation}
where $\sigma^2$ is a variance and $\mu$ is an average.  In the
case of spike trains it is usually applied to the spike count; so to
calculate it you divide the spike train into intervals and work out
the spike count for each interval. The Fano factor is calculated using
the average and variance for these counts. The coefficient of variation is 
\begin{equation}
C_v=\frac{\sigma}{\mu}
\end{equation}
where $\sigma$ is a standard deviation and $\mu$ is an average. In the
case of spike trains this is usually applied to the inter-spike
interval, that is the time difference between successive spikes.

\subsection*{Question 1}

Write code to generate spikes using a Poisson process. Calculate the
Fano factor of the spike count and coefficient of variation of the
inter-spike interval for 1000 seconds of spike train with a firing
rate of 35 Hz, both with no refractory period and with a refractory
period of 5 ms. In the case of the Fano factor the count should be
performed over windows of width 10 ms, 50 ms and 100 ms. [10 marks]

\subsection*{Question 2}

In the \texttt{spike\_trains} folder you will find the data file
\texttt{rho.dat}. This is contains data collected and provided by Rob
de Ruyter van Steveninck from a fly H1 neuron responding to an
approximate white-noise visual motion stimulus. Data were collected
for 20 minutes at a sampling rate of 500 Hz. In the file, rho is a
vector that gives the sequence of spiking events or non-events at the
sampled time, that is, every 2 ms.

Calculate the Fano factor and coefficient of variation for this spike
train as for the simulated spike trains above. [10 marks]


\subsection*{Question 3*}

Consider an integrate and fire neuron, such as the one described in
the formative coursework. With a constant input this neurons spikes
periodically so that the variation in spike intervals is zero and if
there is a variation in spike counts it is just the trivial
consequence of a mismatch in the spike period and the interval
overwhich counting is being performed. However, it is possible add
noise to the input. One way to do this would be to include a synaptic
input where pre-synaptic neurons are modelled by Poisson
processes. Examine what happens in this case, try to keep the firing
rate of the post-synaptic input constant but experiment with lots of
weak Poisson input or a small amount of strong Poisson input. Consider
having both inhibitory and excitatory input. Report on the Fano factor
and coefficient of variation; for the highest marks consider plotting
these against a parameter like the synapse strength or excitatory to
inhibatory ratio. [15 marks]


\subsection*{The spike triggered average}

The spike triggered average is used to identify what is causing the
neuron to spike. Imagine the neuron spikes at times
$\{t_1,t_2,\ldots,t_n\}$ and the stimulus is $\rho(t)$ then the spike
triggered average is
\begin{equation}
s(\tau)=\frac{1}{n}\sum_{i=1}^n \rho(t_i-\tau)
\end{equation}
In other words it is the average value of simulus a time $\tau$ before the
neuron spikes. 

\subsection*{Question 4} 

In the \texttt{spike\_trains} folder you will find the data file
\texttt{stim.dat}. This give the motion stimulus that evoked the spike
train in \texttt{rho.dat}. Calculate and plot the spike triggered
average over a 100 ms window. [10 marks]

\subsection*{Question 5}

Calculate the stimulus triggered by pairs of spikes, that is for
intervals of 2 ms, 10 ms, 20 ms and 50 ms calculate the average
stimulus before a pair of spikes seperated by that interval; do this
for both the case where the spikes are not neccesarily adjacent and
the case where they are. [10 marks]

\subsection*{Question 6*} 

Again, compare your results to an integrate and fire neuron: the idea
is that the integrate and fire would receive an input based on the
stimulus file; to ensure the neuron fires you will need to also
include the direct current, but this can be modified with the addition
of the stimulus. What is the spike triggered average? You can
experiment with changing the stength of the stimulus contribution and
parameters like the membrane constant. [15 marks]

\subsection*{General instruction}

There are 70 marks in total, this will be used to calculate a
percentage. The two questions marked with an asterisk are harder and
more open ended; do not spend too much time on these, the marking will
give some credit for any attempt at the modelling they involve, but to
get most or all of the 15 marks will require some experimentation.

Scientific communication of relies strongly on graphs and so you
should take care in producing nice graphs; marks will be deducted for
graphs where the text is tiny, or relevant units are missed. Do not
include graphs that have been produced by screenshotting.
Neuroscience as a subject favours long figure captions which provide
detailed information about the graphs show, without giving
interpetation, that is left to the main text. Try to follow this
convention.

Please do not exceed six pages, however, you should not regard six
pages as a target length, a much shorter submission could still earn
all or most of the marks. The reason the length is not shorter is to
make it easier to write without having to worry about being very
precise, it is not to indicate that a long response is required. Do
not use a font size less than 11pt and have normal looking
margins. Submission is in pdf. You should include small code extracts
where appropriate, any coding language can be used. You will also be
required to upload your code. Submission is through blackboard upload.

This is a draft of the coursework offered in advance of the coursework
period which starts on Monday 2023-11-20; between now and then I may
make small adjusts to correct typos, to make more precise the
submission instructions and to add some small clarifications. However
the actual tasks will remain the same.


\end{document}
